\documentclass[onecolumn,english,prl,showpacs]{revtex4}
\usepackage[colorlinks=true,urlcolor=blue,citecolor=blue,linkcolor=blue]{hyperref}
\usepackage[sort&compress]{natbib}
\usepackage{ucs}
\usepackage[utf8x]{inputenc}
\usepackage{amssymb}
\usepackage{graphicx}
\usepackage{amsmath,color}
\usepackage{mathrsfs}
\usepackage{float}
\usepackage{indentfirst}
\usepackage{babel}	
\usepackage{bbold}

\newcommand{\rint}{\int\displaylimits}
\renewcommand{\Re}{\mathop{\text{Re}}\nolimits}

\renewcommand{\Im}{\mathop{\text{Im}}\nolimits}
\newcommand{\D}{\mathcal{D}}
\newcommand{\tr}{\mathop{\text{tr}}\nolimits}
\newcommand{\Tr}{\mathop{\text{Tr}}\nolimits}
\newcommand{\ket}[1]{|{#1}\rangle}
\newcommand{\bra}[1]{\langle{#1}|}
\newcommand{\bras}[2]{{}_{#2}\hspace*{-0.2mm}\langle{#1}|}
\newcommand{\bracket}[2]{\langle#1|#2\rangle}
\newcommand{\Det}{\mathop{\text{Det}}\nolimits}
\newcommand{\Res}{\mathop{\text{Res}}}
\newcommand{\pv}{\mathop{\text{P}}\nolimits}
\newcommand{\erf}{\mathop{\text{erf}}\nolimits}
\newcommand{\erfc}{\mathop{\text{erfc}}\nolimits}
\newcommand{\erfi}{\mathop{\text{erfi}}\nolimits}
\newcommand{\sinc}{\mathop{\text{sinc}}\nolimits}
\newcommand{\sech}{\mathop{\text{sech}}\nolimits}
\newcommand{\Ci}{\mathop{\text{Ci}}\nolimits}
\newcommand{\Texp}{\mathop{\text{Texp}}\nolimits}

\newcommand{\p}{^{\vphantom{\dagger}}}
\newcommand{\h}{^{\dagger}}

\newcommand{\Up}{{\uparrow}}
\newcommand{\Dn}{{\downarrow}}

\begin{document}

\title{A continuous version of the BSS quantum Monte Carlo algorithm}

\author{Mauro Iazzi}
\author{Matthias Troyer}

\affiliation{Theoretische Physik, ETH Zurich, 8093 Zurich, Switzerland}

\begin{abstract}
\end{abstract}

\section{Partition function expansion}

\begin{equation}
 Z = \tr\left[ e^{-\beta (H_0+H_I-KV)} \right]
\end{equation}

\begin{multline}
 Z = \tr\left[ e^{-\beta (H_0+H_I-KV)} \right]=\\=
  \tr\left\{e^{-\beta H_0}\sum_n^\infty \frac{(-1)^n}{n!} {\mathcal{T}}\int_0^\beta \prod_{i=1}^n (H_I(\tau_i)-KV) d\tau_n\right\}=\\=
  \tr\left\{e^{-\beta H_0}\sum_n^\infty \frac{K^nV^n}{n!} {\mathcal{T}}\int_0^\beta \prod_{i=1}^n (1-H_I(\tau_i)/KV) d\tau_n\right\}=\\=
  \sum_n^\infty \frac{K^nV^n}{n!} {\mathcal{T}}\int_0^\beta \prod_{i=1}^n d\tau_n \tr\left[e^{-\beta H_0}\prod_{i=1}^n (1-H_I(\tau_i)/KV)\right]=\\=
  \sum_n^\infty {\mathcal{T}}\int_0^\beta \frac{w(\tau_1, ..., \tau_n)}{n!} \prod_{i=1}^n d\tau_n 
\end{multline}
\begin{equation}
 w(\tau_1, ..., \tau_n) = K^nV^n\tr\left[e^{-\beta H_0}\prod_{i=1}^n(1-H_I(\tau_i)/KV)\right]
\end{equation}
\begin{multline}
 1-H_I/KV=1-\frac{g}{KV} \sum_x \hat{n}_{x, \Up} \hat{n}_{x,\Dn} = \frac{g}{KV}\sum_x \left(\frac{K}{g}-\hat{n}_{x, \Up} \hat{n}_{x,\Dn}\right)=\\=
 \frac{g}{2KV}\sum_x \sum_{\sigma_x=\pm1} \left(\sqrt{\frac{K}{g}}+\sigma_x\hat{n}_{x, \Up}\right)\left(\sqrt{\frac{K}{g}}-	\sigma_x \hat{n}_{x,\Dn}\right)=\\=
 \frac{1}{2V}\sum_x \sum_{\sigma_x=\pm1} \left(1+\sqrt{\frac{g}{K}}\sigma_x\hat{n}_{x, \Up}\right)\left(1-\sqrt{\frac{g}{K}}\sigma_x \hat{n}_{x,\Dn}\right)
\end{multline}
\begin{equation}
 w(\tau_1, ..., \tau_n) = (KV)^n\det\left\{1+G(\beta-\tau_n)\prod_{i=1}^n (1+\sqrt{g/K}\sigma_x(\tau_i)\hat{n}_x)G_0(\tau_i-\tau_{i-1})\right\}
\end{equation}
\begin{multline}
 Z = \tr\left\{e^{-\beta H_0}\sum_n^\infty \frac{(-K)^n}{n!} {\mathcal{T}}\int_0^\beta \prod_{i=1}^n (1+H_I(\tau_i)/K) d\tau_n\right\} =\\=
  \tr\left\{e^{-\beta H_0}\sum_n^\infty \frac{(-K)^n}{n!} {\mathcal{T}}\int_0^\beta \prod_{i=1}^n \frac{1}{2V} \sum_{x_i=1}^V\sum_{\sigma_i=\pm1} e^{+\tau_i H_0}\left(1+\sqrt{\frac{gV}{K}}\sigma_x\hat{n}_{x, \Up}\right)\left(1+\sqrt{\frac{gV}{K}}\sigma_x \hat{n}_{x,\Dn}\right) e^{-\tau_i H_0} d\tau_i\right\}=\\=
  \sum_n^\infty \frac{(-K)^n}{n!} \frac{1}{2^nV^n} \int_0^\beta\ldots\int_0^\beta d\tau_1 \ldots d\tau_n \sum_{\{x_i\}}\sum_{\{\sigma_i\}} \tr\left\{e^{-\beta H_0} {\mathcal{T}}\prod_{i=1}^n e^{+\tau_i H_0}\left(1+\sqrt{\frac{gV}{K}}\sigma_x\hat{n}_{x, \Up}\right)\left(1+\sqrt{\frac{gV}{K}}\sigma_x \hat{n}_{x,\Dn}\right) e^{-\tau_i H_0}\right\}
\end{multline}

\begin{equation}
 (1+\sigma \mathbf{v}\mathbf{v^t})^{-1} = 1-\frac{\sigma}{1+\sigma} \mathbf{v}\mathbf{v^t}
\end{equation}
\begin{equation}
 (1-\frac{\sigma}{1+\sigma} \mathbf{v}\mathbf{v^t})^{-1} = 1-\frac{-\frac{\sigma}{1+\sigma}}{1-\frac{\sigma}{1+\sigma}} = 1-\frac{-\sigma}{1+\sigma-\sigma} \mathbf{v}\mathbf{v^t} = 1+\sigma \mathbf{v}\mathbf{v^t}
\end{equation}

Full pivoting LU
\begin{equation}
 G_i = P^{-1} LU Q^{-1} = P^{-1} LU' D Q^{-1} = P^{-1} LU' Q^{-1} Q D Q^{-1}
\end{equation}
$Q D Q^{-1}$ is diagonal may be applied directly to the SVD vector

Single vertex probability
\begin{multline}
 \det [1+e^{\beta\mu}e^{\beta H_0}(1+\sigma vv^t)]\rightarrow\\\rightarrow 1+ \sigma v^t (1+e^{\beta\mu}e^{\beta H_0})^{-1} e^{\beta\mu}e^{\beta H_0} v =\\=
\end{multline}

\begin{equation}
 1 + U D V^t \rightarrow U (U^t + D V^t) = (V + )
\end{equation}

\section{A trick}
One can use a trick to kill degeneracies. Use a random matrix $R$
\begin{equation}
 \det(1+\prod_iG_i) = \det(1+R^{-1}\prod_iG_iR)
\end{equation}
The rank-1 update becomes
\begin{multline}
 \frac{\det(1+R^{-1}\prod_{i=1}^NG_i(G_0+uv^t)R)}{\det(1+R^{-1}\prod_iG_iR)} = \det(1+ v^tR (1+R^{-1}\prod_iG_iR)^{-1}R^{-1}\prod_iG_iG_0^{-1}u ) =\\=
 \det(1+ v^tR \frac{1}{1+R^{-1}\prod_iG_iR}R^{-1}\prod_iG_iG_0^{-1}u ) =\\=
 \det(1+ v^t \frac{1}{1+\prod_iG_i}\prod_iG_iG_0^{-1}u ) =
 \det(1+ v^t \frac{1}{1+\prod_iG_i^{-1}}G_0^{-1}u )
\end{multline}
Which is the usual formula without any $R$ of course. Let's put it back (cause we want to compute stuff with it)
\begin{multline}
 \frac{\det(1+R^{-1}\prod_{i=1}^NG_i(G_0+uv^t)R)}{\det(1+R^{-1}\prod_iG_iR)} =
 \det(1+ v^t \frac{1}{1+\prod_iG_i^{-1}}G_0^{-1}u ) =\\=
 \det(1+ v^t RR^{-1}\frac{1}{1+\prod_iG_i^{-1}}RR^{-1}G_0^{-1}u ) =\\=
 \det(1+ v^t R\frac{1}{1+R^{-1}\prod_iG_i^{-1}R}R^{-1}G_0^{-1}u ) =\\=
\end{multline}

\section{sign}
\begin{equation}
 G_n(1+\sigma)G_{n-1}\ldots G_{2}(1+\sigma)G_1
\end{equation}
\begin{equation}
 (G_n(1+\sigma)G_{n-1}\ldots G_{2}(1+\sigma)G_1)^{-1} = G_1^{-1}(1-\sigma)G_{2}^{-1}\ldots G_{n-1}^{-1}(1-\sigma)G_n^{-1}
\end{equation}

\section{a better representation for the sign}

\begin{multline}
 1+\frac{g}{K} \sum_x (\hat{n}_{x, \Up} \hat{n}_{x,\Dn}-\frac{\hat{n}_{x, \Up}+\hat{n}_{x,\Dn}}{2}) = \frac{1}{V}\sum_x \left(1+\frac{gV}{K}\hat{n}_{x, \Up} \hat{n}_{x,\Dn}-gV\frac{\hat{n}_{x, \Up}+\hat{n}_{x,\Dn}}{2K}\right)=\\=
 \frac{1}{V}\sum_x \frac12 \sum_{\sigma_x=\pm1} \left[1+(a+b\sigma_x)\hat{n}_{x, \Up}\right]\left[1+(a+b\sigma_x) \hat{n}_{x,\Dn}\right]=\\=
 \frac{1}{V}\sum_x \frac12 \sum_{\sigma_x=\pm1} \left(1+(a+b\sigma_x)\hat{n}_{x, \Up}+(a+b\sigma_x) \hat{n}_{x,\Dn} + (a+b\sigma_x)^2\hat{n}_{x, \Up} \hat{n}_{x,\Dn} \right)
\end{multline}
\begin{eqnarray}
 a = -\frac{gV}{2K},\\
 a^2+b^2 = \frac{gV}{K}.
\end{eqnarray}
\begin{eqnarray}
 a = -\frac{gV}{2K},\\
 b = \pm\sqrt{\frac{gV}{K}-\frac{g^2V^2}{4K^2}}
\end{eqnarray}

Now since $a^2+b^2+2a=0$ then $1+a+b = (1+a-b)^{-1}$ which means that absent a magnetic field the model is sign-problem free.

If $K=\frac{gV}{4}$ b=0 and we have LCT-INT

\section{fast updates}

\begin{equation}
 G = \frac{1}{1+B^{-1}}
\end{equation}

\begin{equation}
 B \rightarrow (1+uv^t)B
\end{equation}

\begin{equation}
 B^{-1} \rightarrow B^{-1}(1+uv^t)^{-1}
\end{equation}

\begin{equation}
 \frac{\det(1+(1+uv^t)B)}{\det(1+B)} = \frac{\det(B)}{\det(B)}\cdot \frac{\det(B^{-1}+1+uv^t)}{\det(B^{-1}+1)} = 1+v^t\frac{1}{1+B^{-1}} u = 1+v^t G u
\end{equation}

\begin{equation}
 B \rightarrow (1+uv^t)B
\end{equation}

\begin{multline}
 G \rightarrow \frac{(1+uv^t)B}{1+(1+uv^t)B} = (1+uv^t)\frac{1}{B^{-1}+1+uv^t} =\\= (1+uv^t)\left[\frac{1}{B^{-1}+1}-\frac{1}{B^{-1}+1}\cdot\frac{uv^t}{1+v^t\frac{1}{B^{-1}+1}u}\cdot\frac{1}{B^{-1}+1}\right]=\\= (1+uv^t)\left[G-G\cdot\frac{uv^t}{1+v^tGu}\cdot G\right]
\end{multline}


\end{document}
